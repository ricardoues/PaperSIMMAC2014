\documentclass[11pt,twoside]{rmta2010esp}% For English, use rmta2010eng.cls instead of rmta2010esp.cls
%\pagestyle{myheadings} 
\usepackage[english, spanish]{babel}
\usepackage{fancyhdr}
\usepackage[utf8]{inputenc} % En caso de usar tildes, y otros caracteres especiales



\pagestyle{fancy}
\fancyhead{}
\fancyhead[RE]{\thepage \hfill {\sc a. rib\'{o} -- g. molina -- r. r\'{i}os  }}
\fancyhead[LO]{{\sc Combining neural networks and geostatistics for landslide hazard assessment of San Salvador Metropolitan Area, El Salvador } \hfill \thepage}
\fancyfoot{}
\fancyfoot[RE]{{\scriptsize{\it Rev.Mate.Teor.Aplic.} (ISSN 1409-2433) {Vol. 19}(1): 1--10, January 2012}}
\fancyfoot[LO]{{\scriptsize{\it Rev.Mate.Teor.Aplic.} (ISSN 1409-2433) {Vol. 19}(1): 1--10, January 2012}}
\paginas{1}{10} % Numeros de pgina, la versión final la pone el editor; Page numbers, the Editor puts the final numbers
\usepackage{amsmath}
\usepackage{amssymb}
\usepackage{graphics}
\usepackage{graphicx} % En caso de usar figuras, con formato .esp; In case of using figures, with format .eps
\usepackage{float} 
\usepackage{url}
\usepackage{amsfonts}
\usepackage{amsmath}
\usepackage{enumerate}
\usepackage{authblk}
\usepackage{hyperref}


\DeclareGraphicsExtensions{.bmp,.png,.pdf,.jpg,.eps}



\renewcommand{\refname}{References} 



\begin{document}

\selectlanguage{spanish}
\title{Combining neural networks and geostatistics for landslide hazard assessment of San Salvador Metropolitan Area, El Salvador 
\newline
\newline
Combinando redes neuronales y geoestad\'{i}stica para evaluaci\'{o}n de deslizamientos de tierra de el \'{a}rea Metropolitana de San Salvador, El Salvador}

\author[1]{Ricardo R\'{i}os \thanks{ricardo.sv@gmail.com}}
\author[2]{Alexandre Rib\'{o} \thanks{alexandre4rt@gmail.com}}
\author[3]{Roberto Mej\'{i}a \thanks{robertomejia1685@gmail.com}}
\author[4]{Giovanni Molina \thanks{giova.molina@gmail.com}}

\affil[1]{Department of Mathematics, Science and Mathematics Faculty, University of El Salvador, El Salvador.}
\affil[2]{National Institute of Health, Ministry of Health of El Salvador, El Salvador.}
\affil[3]{National Institute of Health, Ministry of Health of El Salvador, El Salvador.}
\affil[4]{Ministry of Environment and Natural Resources of El Salvador.}



\maketitle

%\newpage 

\selectlanguage{spanish}
\begin{resumen}
Esta contribución describe la creaci\'{o}n de un modelo de evaluaci\'{o}n de deslizamiento de tierra para San Salvador, departamento de El Salvador. El an\'{a}lisis inicio con la obtenci\'{o}n de una foto \'{a}rea del MARN (Ministerio de Medio Ambiente y Recursos Naturales) con un total de 939407 puntos georeferenciados con el fin de producir un inventario de deslizamiento. En esta evaluaci\'{o}n de los deslizamientos se uso 4792 eventos previamente foto-interpretados y 7 factores condicionantes incluyendo: geomorfolog\'{i}a, geolog\'{i}a, precipitaciones m\'{a}ximas, aceleraciones s\'{i}smicas, pendiente del terreno, distancia a carretera y falla geol\'{o}gica. Redes Neuronales Artificiales (RNA) fueron usadas para la evaluaci\'{o}n de la susceptibilidad a deslizamiento de tierra, logrando que m\'{a}s del 80\% de deslizamientos fueran apropiadamente clasificados usando un criterio dentro y fuera de la muestra con la que se estimaron los parámetros del modelo. Regresi\'{o}n Log\'{i}stica fue usada como base de comparaci\'{o}n, obteniendo este modelo un rendimiento inferior que el de RNA con un porcentaje de correcta clasificaci\'{o}n abajo del 70\%. Para completar el an\'{a}lisis se realizo la interpolaci\'{o}n de puntos usando el m\'{e}todo kriging proveniente del enfoque geoestad\'{i}stico. Finalmente, los resultados muestran que es posible obtener un mapa de riesgo a deslizamiento de tierra, haciendo uso de una combinación de RNA y t\'{e}cnicas geoestad\'{i}sticas con lo cual la presente investigaci\'{o}n puede ayudar a la mitigaci\'{o}n de deslizamientos de tierra en El Salvador.
\end{resumen}

\PC deslizamiento de tierra, evaluación de riesgo, El Salvador, RNA, geoestadística.

\begin{abstract}
This contribution describes the creation of a landslide hazard
assessment model for San Salvador, department in El Salvador. The analysis started with an aerial photointerpretation from  MARN (Ministry of Environment and Natural Resources of El Salvador)  with a total amount of 939407 georeferenced points to produce a landslide inventory. In this
landslide assessment we have used 4792 events previously photo-interpretaded
and 7 conditioning factors including: geomorphology, geology, rainfall intensity, peak ground accelaration, slope angle, road and fault distance. Artificial Neural Networks (ANN) were applied for the assessment of susceptibility to
landslides, achieving more than 80\% of landslide were properly classified using
in-sample and out of sample criteria. Logistic regression was used as base of
comparison, obtaining this model a performance lower than ANNs with a percentage of correct classification under 70\%. To complete the analysis we have
performed interpolation of the points using kriging method from geostatistical
approach. Finally, the results show that is possible to derive a landslide hazard map, making use of a combination of ANNs and geostatistical techniques
wherewith the present study can help landslide mitigation in El Salvador.
\end{abstract}

\KW landslide, hazard assessment, El Salvador, ANN, geostatistics.

\AMS 62P12.% Mathematical Subject Classification, http://www.ams.org/mathscinet/msc/msc2010.html.

\selectlanguage{english}


\section{Introduction}
\label{sec:intr}
El Salvador, one of the smallest and most crowded nations in Central America, extends in Pacific coast about 240 kilometers westward from the Gulf of Fonseca to the border with Guatemala (see Fig \ref{fig:mass01}). El Salvador borders an active subduction boundary between Cocos and Caribbean plates located 30 km offshore. Therefore El Salvador is affected with high seismicity and volcanic activity related to this active boundary. There is two main sources of seismic activity, the upper interface thrust that coincides with the location of the recent volcanoes and the Benioff-Wadaty zones of subducted Cocos plate where deeper intraplate earthquakes occurs (\cite{dewey}). Due this tectonic and volcanic activity El Salvador has rugged relief. The main ranges cross the country with a rough west-east trend, parallel to the coast, these are separated from each other by faults and grabens. These ranges present several highly active volcanoes. The surface geology in El Salvador is almost entirely volcanic, dominated by upper Tertiary to Holocene volcanic rocks, only sparse outcrops of sedimentary and plutonic crops are located in the northern ranges, in the border with Honduras (\cite{weber}). Some of the most recent volcanic layers are formed by poor consolidated ashes and tuffs highly erodible (\cite{bommer}).

Throughout the year the country experiences a tropical climate with two seasons, a dry season (November to April) and wet season (May - October). The climate of El Salvador is generally warm. In the dry season there is very less rainfall but during rainy seasons heavy showers take place. El Salvador's interior regions remain dry throughout the year. Periodically El Salvador is affected by tropical storms and occasionally by hurricanes.

Like in other parts of Central America, landslides in El Salvador constitute an important natural hazard due to prevailing steep terrain covered with poor consolidated volcanic materials and the frequent occurrence of extreme precipitation events and intense earthquakes. This problem is exacerbated by the extreme deforestation and the consequent high level of rates of erosion. Poverty, overpopulation and uncontrolled urbanization that characterizes the Salvadoran human settlements converts El Salvador in a country with high landslide risks.
One example of this high risk is the devastating effect of the Las Colinas landslide triggered by a major earthquake (Mw 7.6) occurred on January of the 13, 2001  in Santa Tecla, a major city located close to San Salvador  the Salvadoran capital (\cite{evans}). A huge amount of soil mass (about 200,000 m3) was thrown off the rim of El Balsamo range and flushed many houses and produced more than 500 deaths. Together with this event this earthquake triggered several landslides along the country, especially in the Metropolitan Area of San Salvador (MASS) (\cite{jibson}). Other example is the large number of heavy-rainfall induced landslides occurred during Hurricane Mitch on October and November 1998 (\cite{crone}).   

Due to the above it is necessary to implement mechanisms that allow us to quantify the hazard of a given geographic area to landslides, usually this is done with development of susceptibility maps which present in a graphical way the zones more susceptible to landslides and represent a practical tool for urban planning. It was proposed an Artificial Neural Network model (ANN) which are a family of statistical learning models inspired by biological neural networks. The model proposed is used to estimate the susceptibility to landslide. From the results obtained by the model, a map was derived using kriging which is a method of interpolation spatial from the geostatistical approach. 


 \begin{center}
%  \begin{figure}[H]
  \begin{figure}
   \centering
   \includegraphics[scale=0.75]{MASS_mapa_1}
   \caption{\small{Maps from El Salvador}}
   \label{fig:mass01}
  \end{figure}
 \end{center}




\section{Brief State of Art}
\label{sec:brief}
Since the pioneering work (\cite{Carrara1983403}), several mathematics and statistics models have been proposed to model landslide susceptibility: deterministic models (\cite{hessd-10-12643-2013},  \cite{doi:10.1080/19475705.2010.498151} and \cite{Neu2012511}), probabilistic models (\cite{Bern198839}, \cite{Chung2003451} and \cite{doi:10.1080/01431160310001618734} ). 

It has been used popular classification models such as logistic regression (\cite{akgun2012} and \cite{gaskill} and \cite{garcia2008} ) , neural networks (\cite{Melchiorre2011410},\cite{Zeng2001374}, \cite{Ermini2005327}) and \cite{Yesilnacar2005251} and support vector machine (\cite{ballabio2012support} and \cite{tien2012landslide}). 


According to (\cite{van2006landslide}) the magnitude of a possible
slide is difficult to foresee as it depends on the magnitude of the triggering event and the environmental conditions (e.g., height of water table) at the moment of the event. Because of these complex relationships between the dependent variable and causal factors, and since that neural networks are particularly useful for detect complex non-linear relationships in large datasets, that is why this kind of model was chosen, despite the disadvantages such as greater computational burden and proneness to overfitting.  



%There are two important applications of the neural network models to landslide susceptibility. The first (\cite{Yesilnacar2005251})

%Despite of the recent use of maps in those published works have not given enough attention on the use of geostatistical models to interpolate spatially distributed landslide susceptibility just landslide susceptibility is given in the georeferenced points and showed in the map of susceptibility.

\section{Study area}
\label{sec:studyarea}
The Metropolitan Area of San Salvador (MASS), is formed by 14 municipalities among them there is San Salvador, the capital of the country. It has 587 km2 and according to the most recent census (\cite{minecon}) it has 1,566,629 inhabitans with a density of population of 2669 hab/km2.  According to geography MASS is extended over a flat erosional surface (650-760 m a.s.l.), gently sloping to the east towards Ilopango Caldera also known as Ilopango Lake. MASS is bordered to the south by the Balsamo Range and San Jacinto Hill, to the west by San Salvador volcano and with Mariona Hills (refer to Fig \ref{fig:mass01}). Human settlement is not only restricted to the plain, but spreads up to surrounding heights and volcano flanks.

All rocks exposed in MASS have volcanic origin and consists of intercaled primary and reworked deposits from Late Tertiary to Holocene age (\cite{schmidt1975}). The younger ones are known as Tierra Blanca pyroclastic ash deposits and forms a relative continuous layer in majority of MASS with an average thickness of 4 m (\cite{schmidt1975}). These are deposits derived mainly from Ilopango caldera eruptions where they present more than 50 m of thickness (\cite{schmidt1975}). Tierra Blanca soils is highly susceptible to erosion, one of the most common dangerous effects of the erosion in this area are landslides during heavy rainfall and earthquake ground shaking (\cite{schmidt1975}). (\cite{chavez2014a}) identifies high and moderate erosion hazard in majority of MASS area. The more recent layer Tierra Blanca, known as Tierra Blanca Joven (TBJ) (\cite{hernan2004}) present very poor geotechnical properties, especially when there is an increment of soil moisture (i.e. during rainy seasons), which results in high instability susceptibility (\cite{chavez2014b} and \cite{rolo2004}). Several incised rivers and ravines cross MASS area, due the large erosion rates of the area, natural slopes of these watercurses are often close to vertical and can reach highs of more than 10 m. Taking into account this property of volcanic soils, usually vertical slopes are cut to urban and road construction. As (\cite{bommer}) identify in Central America, although such slopes may remain stable for years, they may become unstable, abruptly and totally under the action of heavy rainfall or seismic shaking. (\cite{hernan2004}) carry out a detailed description of this process in TBJ, where these instabilities are very common. 
In study area fault trends are characterized by steep dip angles (650-900). The main fault system is characterized by an east-west trend; this trend is responsible of the steep northern slope of the El Balsamo Range.  The secondary systems are with north, northwest and northeast trends offset the main fault system and are more active in the present tectonic setting (\cite{schmidt1975}). However, all the mentioned fault families appear active. Thus the several fault systems have developed at different times, but been repeatedly activated (\cite{rymer1987}).   Finally there are also several ringlike faults that can be related to subsurface collapse due to volcanic tectonic subsidence (\cite{schmidt1975}).
Last eruption of San Salvador volcano was 1917.  The majority of the magma's total volume (97\%) extruded from several vents in the Northwest flank of the volcano, outside of MASS. However a fraction of MASS is located under the proximal volcanic hazard zone according to (\cite{sofield2004}). In the steep slopes  San Salvador volcano, especially of Picacho, the highest peak of the volcano, located in the eastern side present a high lahar hazard that threats MASS (\cite{major2004}). On the other hand one of the most destructive eruption in in the world was 430 A.D. TBJ eruption produced by the Ilopango Caldera. According to (\cite{dull2004}), that eruption produced total devastation in whole MASS area. 




\section{Landslide inventory and data sources of input variables}
\label{sec:landsinvet}
The Landslide inventory was developed by MARN and it was conducted in two stages. The first was carried out, based on fieldwork after the 2001 eartquakes in El Salvador. The last stage is complete with a visual inventory of places where there were landslides in visible satellite image IRSC (India) sensor of the same year.


from photographic analysis on the study area, using aerial photo from MARN which is presented in the Figure \ref{fig:img01}, where white areas represent regions of landslide occurrence which were processed using the software ILWIS. 

% \begin{center}
%  \begin{figure}[H]
%   \centering
%   \includegraphics[scale=0.40]{img01}
%   \caption{\small{Aerial photo of MASS}}
%   \label{fig:img01}
%  \end{figure}
% \end{center}

Once that the white areas were georeferenced, a percentage of 0.5\% of the total points georeferenced shows landslide ocurrence. In addition to the landslide information the following data sources of input variables was given by MARN: 
\begin{enumerate}
\item {\bf Geomorphology:} refers to landforms that result from lithospheric dynamics of geographic area.

\item {\bf Slope:} derived from digital elevation model MARN

\item {\bf Geology:} Description of the geology of the area from the map German Geological Mission (scale 1:100,000).

\item {\bf Rainfall intensity:} Maximum rainfall recorded in the geographical area.

\item {\bf Peak ground accelaration:} Maximum ground acceleration expressed in Gal for a return period of 500 years, this is the least that has detailed information. This information was obtained from RSIS II project.

\item {\bf Road distance:} Distance in kilometers to the nearest road. 

\item {\bf Fault distance:} Distance in kilometers to the nearest fault. This information was obtained from German Geological Mission (scale 1:100,000). 
 
\end{enumerate}


%%% We must comment page 12 - 14 from my Master Thesis 
%%% Valorizar si lo de Redes Neuronales y Predicción Espacial lo ponemos en métodos y materiales 

\section{Artificial Neural Network model for discrete choice}
The logistic regression is a special case of neural network whose output variable is
discrete, logistic regression represents a neural network with 
a neuron in the hidden layer. The following adaptation of a multilayeredfeedforward artificial neural network known as  
MLP (Multilayer Perceptron) may be used for modeling binary classification model, where $x's$ are the observed values in the input variables, $w's$ and $\lambda's$ are the parameters of the model, $ p_{i} $ is the predicting probability for a network with $ k^{*} $ input characteristics and $ j^{*} $ neurons: 

\begin{equation}
n_{j,i} = w_{j,0} + \sum_{k=1}^{k^{*}} w_{j,k}x_{k,i}
\end{equation}

\begin{equation}
N_{j,i} = \frac{1}{1+\exp^{-n_{j,i}}}
\end{equation}



\begin{equation}
p_{i} = \sum_{j=1}^{j^{*}} \lambda_{j} N_{j,i}
\end{equation}

\begin{equation}
\sum_{j=1}^{j^{*}} \lambda_{j} = 1 , \lambda_{j} \ge 0
\end{equation}



In the context of the research problem $p_{i} $ and the number $ k^{*} $ represent 
probability of landslide ocurrence and the number of input variables or causes associated with landslide ocurrence respectively.

Before estimating the parameters of the neural network model, it is necessary standardize the input variables. In particular for classification problems is more suitable to scale inputs to $[-1,1]$ rather than $[0,1]$ (\cite{FAQANN}). The following scaling was applied to each input variable: 

\begin{equation}
x*_{k,i} = \frac{x_{k,i} - \mu }{\sigma}
\end{equation}

Where $ \mu $ and $\sigma$  is respectively the mean and the standard deviation for the ith input variable applied to the kth case.  

The method used for estimating the parameters of the model was an Hybrid Method (\cite{McNelis2005}): Firstly heuristic genetic algorithm using a package developed in the R statistical software specifically developed for this purpose, was used to obtain 
a good estimation of the parameters of the model. 

The R package can be accessed from the following web address:
 
\url{https://goo.gl/PHaaG2}


Once a good estimate was obtained, this was occupied as the initial values for the conjugate gradient method implemented in the optim function of the R statistical software, to obtain a better estimation of the parameters of the model.  


\section{Spatial Prediction}
In standard statistical problems, correlation can be estimated from a scatterplot, when several data pairs ${x, y}$ are available. The spatial correlation between two observations of a variable $z(s)$ at locations $s_{1}$ and $s_{2}$ cannot be estimated, as only a single pair is available. To estimate spatial correlation
from observational data, we therefore need to make stationarity assumptions
before we can make any progress. One commonly used form of stationarity
is intrinsic stationarity, which assumes that the process that generated the
samples is a random function $Z(s)$ composed of a mean and residual:

\begin{equation}
Z(s) = \mu + \delta(s)
\end{equation}

with a constant mean 

\begin{equation}
E\left(Z(s)\right) = \mu
\end{equation}

and a variogram defined as 

\begin{equation}
\lambda(h) = \frac{1}{2}E\left(Z(s) - Z(s+h)\right)^{2}
\end{equation}


\subsection*{Ordinary kriging in terms of the covariance function}
The predictor assumption is 
\begin{equation}
\hat{Z(s_{0})} = \sum_{i=1}^{n} w_{i}Z(s_{i})
\end{equation}

It is a weighted average of the sample values, and $ \sum_{i=1}^{n} w_{i} = 1 $ to ensure unbiasedness. The $w_{i}$'s are the weights that will be estimated. 
  
Kriging minimizes the mean squared error of prediction 

\begin{equation}
min \ \sigma_{e}^{2} = E\left[Z(s_{0}) - \hat{Z(s_{0})}\right]^{2}
\end{equation}  

In order to make spatial predictions using ordinary kriging, an R script was developed which can be accessed in the following web address: 



\section{Results}
%\subsection*{Data Partition}
The data were randomly divided into three sub-samples, the first was used for the
estimation of the parameters of the neural network (training set), the second was used for
choosing the model with more generalization capability (validation set) and the last was
used to evaluate how well the model generalize outside of the data set used for estimation (test set). 

To determine the number of neurons in the hidden layer was used the method of trial and error, starting with few neurons and increasing progressively the number of neurons in the hidden layer. Table \ref{annresults}
summarizes how well the models fit on the training, validation and test sets.

\begin{table}[H]
\caption{Summary of classification accuracy on the training, validation and test sets for neural network models  }
\label{annresults}
\centering
\begin{tabular}{ | c | c | c | c | }
\hline
\multicolumn{4}{| c |}{Percentage score} \\
\hline
Número de neuronas &    Train set    &   Validation set &  Test set \\
\hline
2  &  0.7011 & 0.7020 & 0.7124 \\
\hline
3 &  0.7206 & 0.7072 & 0.7401 \\
\hline 
4 &  0.7425 & 0.7197 & 0.7458 \\
\hline
5 &  0.7601 & 0.7604 & 0.7740 \\
\hline
6 &  0.7823 & 0.7704 & 0.7604 \\
\hline
7 &  0.7853  & 0.7704 &  0.7763 \\
\hline
8 &  0.7942 & 0.7865 & 0.7878  \\
\hline
9 & 0.8009 & 0.7871 & 0.7901  \\
\hline
10 & 0.8321 & 0.8132 & 0.8009  \\
\hline
11 & 0.8194 & 0.7792 & 0.7542 \\
\hline
12  &  0.8230 & 0.8006 & 0.8159 \\
\hline
13 & 0.8408 &  0.8006 & 0.8031 \\
\hline
14  &  0.8373 & 0.8017 & 0.8127 \\
\hline
15 &  0.8517 & 0.8210 & 0.8247 \\ 
\hline
16  &  0.8246 & 0.8017 & 0.8028 \\ 
\hline
17  &  0.8543 & 0.8283 & 0.8246 \\ 
\hline
18   &  0.8467 & 0.8022 & 0.8418 \\ 
\hline
19  &  0.8341  & 0.7991 & 0.8274 \\
\hline
\end{tabular}
\end{table}


The model with 17 neurons in the hidden layer was chosen, after that a logistic regression model was fitted as a basis of comparison. Table \ref{logisticresults} presents the logistic regression fits, clearly the logistic regression's performance was lower than the worst neural network model. 

\begin{table}[H]
\caption{Summary of classification accuracy on the training, validation and test sets for logistic regression}
\label{logisticresults}
\centering
\begin{tabular}{ | c | c | c | c | }
\hline
\multicolumn{4}{| c |}{Percentage score} \\
\hline
Model &    Train set    &   Validation set &  Test set \\
\hline
logistic regression  &  0.6710  & 0.6748   & 0.663 \\
\hline
\end{tabular}
\end{table}
 
To generate the map of landslide hazards, the following steps were followed based on geostatistics methodology: Exploratory analysis, Variogram modelling and Spatial prediction using ordinary kriging and validation of this results. 

In order to achieve the above steps a R scrit was developed which can be downloaded in the following address:

 

First of all one thousand fifty georeferenced point with their probability of ocurrence which was calculated using neural network, were selected randomly for the map generation process leaving the others for validation of the model. The Exploratory analysis showed the presence of spatial correlation in the data, this was done making scatter plot of pairs $Z(s_{i})$ and  $Z(s_{j})$, grouped according to their separation distance, also the data did not show the presence of anisotropic effect. 

As regards the estimation of the variogram, the spherical, gaussian and exponential models were fitted but the validation of these models on data that were not taken into account in the process of estimating the variogram showed that the best model was the exponential, since that it showed a $ R^{2} = 0.42 $ against $ R^{2} = 0.40 $ and $ R^{2} = 0.39 $ of the models spherical and gaussian respectively. With the exponential model were made spatial predictions on a grid of 1500 $\times$ 1500 and after that, the spatial predictions were imported to Geotiff raster format with the purpose of using this raster map in ArcGIS to generate the landslide hazard map for MASS. 

% Falta poner que despues de pasarlo al geotiff se utilizo el software 
% ILWIS o el que se tuvo que usar y que posteriormente se genero el 
% map of landslide hazard mostrado en la figura  
% valorizar si se pone el codigo fuente 

%The map of landslide hazards is showed in the figure \ref{fig:img02}.
%
% \begin{center}
%  \begin{figure}[H]
%   \centering
%   \includegraphics[scale=0.40]{img02}
%   \caption{\small{Map of landslide hazards}}
%   \label{fig:img02}
%  \end{figure}
% \end{center}


\section{Discussion and conclusions}
A landslide hazard assessment study was carried out in Metropolitan Area of San Salvador (MASS). The study started with the construction of a landslide inventory and analysis of the causal factors related to the occurrences of landslide. The problem of modelling landslide generated by different causes is very complex and for this reason the study proved the efficacy of the neural network model with a percentage of correct classification around 80\% against other models such as logistic regression with a percentage of correct classification under 70\%. 

In the process of estimating the weights of the model an heuristic technique was used to obtain a better solution after that a local search was used. It is better than use only backpropagation algorithm or another local search method to estimate weights since that when these are used there is a strong danger of getting stuck in a local minimum rather than a global minimum for a vector of weights.

The problem with the neural network approach is that is difficult to estimate the weights of the model due to intensive computation involved in the present study the estimation of a neural network model takes between 4 and 10 hours for this reason we could not assess the statistical significance of the input variables in the neural network processes using boostrapping. For all of the above parallel computing must be used rather than serial computing in estimating weights of neural networks models.


%%% Alex's Discussion 

The results obtained in the geostatistics methodology showed that the spatial the data satisfies the conditions for applying kriging method. Final map of landslide probability of occurrence (see Fig \ref{fig:mass02} ) generated by kriging method is the result of multidisciplinary work which implied an intense geological analysis (i.e. MARN inventory of more than 4000 landslides) together with the use of sophisticated statistical techniques. This landslide hazard map is a user-friendly tool to evaluate landslide risk in MASS and can be useful for many purposes such as territorial planning, prevention and landslide mitigation and so on. 


\begin{center}
%  \begin{figure}[H]
  \begin{figure}
   \centering
   \includegraphics[scale=0.75]{MASS_mapa_2}
   \caption{\small{Landslide hazard map MASS}}
   \label{fig:mass02}
  \end{figure}
 \end{center}



High probability of landslide occurrence in Picacho slope implies very high risk due the proximity of urban areas and due the large volume of mobilized materials.(\cite{major2004}) identified in San Salvador volcano that lahars as voluminous as 2 million of m3 can affect large zones of urban areas. Southern areas of MASS also implies a high risk because large landslides, similar to Las Colinas landslide which involved a total volume of around 150,000 m3 (\cite{evans}), can occur in northern slope of Balsamo range and San Jacinto Hill affecting very crowded urban areas.




The present hazard landslide map is consistent with former landslide hazard maps carried out in the region using approaches conceptually different. (\cite{fernan2008}) presented a landslide hazard GIS map of MASS generated through bivariate statistical method considering several conditioning factors showing similar hazard areas. The only differences are located mainly in the Nejapa Hill, San Jacinto Hill where (\cite{fernan2008}) identifies a slightly higher hazard and the area between San Jacinto Hill and Ilopango Lake where the former study presents a hazard slightly lower. As landslides are an erosional phenomenon, the highest probability areas of landslide occurrence coincides with the areas with highest hazard of erosion in the erosion hazard map of MASS proposed by (\cite{chavez2014a}). There are similarities with susceptibility hazard maps at national scale generated through logistic regression method (\cite{garcia2008}), through artificial neural network method (\cite{garcia2010}) and Mora-Vahrson method (\cite{snet2004}). However the high landslide areas are more extended in these national maps than the present study. These differences could be explained by the working scale, our map is more detailed because is at regional scale. Other possible explanation can be related to the data, present work was built using a more extended landslide catalog and more updated data (e.g. seismic hazard from \cite{beni2012}) than former works.

Rock falls were not included in the catalog; these were very common during 2001 earthquakes in El Balsamo Range (\cite{jibson}). On the other hand, the MASS seismic microzonation of seismic site effects will improve the estimation of peak ground acceleration. The local geologic conditions are responsible for the modifications (amplification, frequency content and duration) experimented by seismic motions just before reaching the ground surface, additionally, the geometry of the ground surface also may produce amplification (topographic effects) (\cite{aki1993}). These effects can be important in soft soils close to incised rivers and also in San Salvador volcano, Balsamo ranges and hills. (\cite{crosta2005}) identified an amplification factor of 1.3-1.4 in the Las Colinas failure area (Balsamo Ranges). All these variables can improve the predictive ability of the model and surely the likelihood of landslides.






%%%%  Mencionar abajo las limitaciones computacionales y 
%%%% que no se pudo tomar una muestra mas grande que 1500 
%%%% para el kriging relacionar con lo del big data que puede ser 
%%%% linea de investigación 

%The results obtained in the geostatistics methodology showed that the spatial the data satisfies the conditions for applying kriging method. The map of landslide hazard \ref{fig:img02} was generated by kriging method and can be used by non-experts in the landslide phenomenon for many purposes such as territorial planning, prevention and mitigation of natural dissasters and so on. 
%
%Other important variables could not be obtained such as landslide types (Rockfalls, Topples, Slides, etc) and temporal information which can improve the predictive ability of the model. 


\section{Acknowledgment}   
The authors are indebted with MARN who provided the data and GIS software such as ILWIS and ARCGIS.




\begin{thebibliography}{99}




\bibitem{dewey}
Dewey J.W. White R.A., Hern\'{a}ndez D.A.
\newblock Seismicity and tectonics of El Salvador
\newblock \emph{Chapter 27 in Geological Society of America Special paper 375 in Natural Hazards in El Salvador of Rose W.I., Bommer J.J., L\'{o}pez D.L., Carr M.J. and Major J.J. 367-378. (2004)}


\bibitem{weber}
Weber H.S., Wiesemann G., Lorenz  W., Schmidt-Thom\'{e} M.
\newblock Mapa geol\'{o}gico de Republica de El Salvador
\newblock escala 1 : 100 000 (6 sheets).
\newblock \emph{ Bundesanstalt f\"ur Geowissenschaften und Rohstoffe, Hannover, Germany. }


\bibitem{bommer}
Bommer J.J., Rodriguez C.E.
\newblock Earthquake-induced landslides in Central Am\'{e}rica
\newblock \emph{Engineering Geology}.
(63):\penalty0 1889-220, 2002


\bibitem{evans}
Evans S.G. Bent A.L.
\newblock Las Colinas landslide, Santa Tecla:  A higly destructive flowslide triggered by January 13, 2001, El Salvador earthquake.
\newblock \emph{Chapter 3 in Geological Society of America Special paper 375 in Natural Hazards in El Salvador of Rose W.I., Bommer J.J., L\'{o}pez D.L., Carr M.J. and Major J.J. 25-37. (2004)}



\bibitem{jibson}
Jibson R.W., Crone A.J., Harp E.L., Baum R.L., Major J.J., Pullinguer C.R., Escobar D., Martinez M., Smith M.E. 
\newblock Landslides triggered by the 13 January and 13 February 2001, earthquakes.
\newblock \emph{ Chapter 6 in Geological Society of America Special paper 375 in Natural Hazards in El Salvador of Rose W.I., Bommer J.J., L\'{o}pez D.L., Carr M.J. and Major J.J. 69-88. (2004)}



\bibitem{crone}
Crone A.J., Baum R.L., Lidke D.J., Sather D.N.D., Bradley L.-A., Tarr A.C.
\newblock Landslides Induced by Hurricane Mitch in El Salvador-An Inventory and Description of Selected Features
\newblock \emph{USGS Open File Repord 01-444. (2002)}



%\bibitem[Carrara(1983)]{Carrara1983403}
\bibitem{Carrara1983403}
Alberto Carrara.
\newblock Multivariate models for landslide hazard evaluation.
\newblock \emph{Journal of the International Association for Mathematical
  Geology}, 15\penalty0 (3):\penalty0 403--426, 1983.
\newblock ISSN 0020-5958.
%\newblock \doi{10.1007/BF01031290}.
\newblock URL \url{http://dx.doi.org/10.1007/BF01031290}.


%\bibitem[Capparelli and P.~Versace(2013)]{hessd-10-12643-2013}
\bibitem{hessd-10-12643-2013}
G.~Capparelli and P.~P.~Versace.
\newblock Landslide susceptibility from mathematical model in sarno area.
\newblock \emph{Hydrology and Earth System Sciences Discussions}, 10\penalty0
  (10):\penalty0 12643--12662, 2013.
%\newblock \doi{10.5194/hessd-10-12643-2013}.
\newblock URL
  \url{http://www.hydrol-earth-syst-sci-discuss.net/10/12643/2013/}.


%\bibitem[Pradhan et~al.(2010)Pradhan, Oh, and
%  Buchroithner]{doi:10.1080/19475705.2010.498151}
\bibitem{doi:10.1080/19475705.2010.498151}
Biswajeet Pradhan, Hyun-Joo Oh, and Manfred Buchroithner.
\newblock Weights-of-evidence model applied to landslide susceptibility mapping
  in a tropical hilly area.
\newblock \emph{Geomatics, Natural Hazards and Risk}, 1\penalty0 (3):\penalty0
  199--223, 2010.
%\newblock \doi{10.1080/19475705.2010.498151}.
\newblock URL
  \url{http://www.tandfonline.com/doi/abs/10.1080/19475705.2010.498151}.
  
  
%\bibitem[Neuhäuser et~al.(2012)Neuhäuser, Damm, and Terhorst]{Neu2012511}
\bibitem{Neu2012511}
Bettina Neuhäuser, Bodo Damm, and Birgit Terhorst.
\newblock Gis-based assessment of landslide susceptibility on the base of the
  weights-of-evidence model.
\newblock \emph{Landslides}, 9\penalty0 (4):\penalty0 511--528, 2012.
\newblock ISSN 1612-510X.
%\newblock \doi{10.1007/s10346-011-0305-5}.
\newblock URL \url{http://dx.doi.org/10.1007/s10346-011-0305-5}.



%\bibitem[Bernknopf and Shapiro(1988)]{Bern198839}
\bibitem{Bern198839}
Campbell R.H. Brookshire~D.S. Bernknopf, R.L. and C.D Shapiro.
\newblock A probabilistic approach to landslide hazard mapping in cincinnati,
  ohio, with application for economic evaluation.
\newblock \emph{Bulletin of the Association of Engineering Geologists},
  XXV\penalty0 (1):\penalty0 39--56, 1988.
  
  
%\bibitem[Chung and Fabbri(2003)]{Chung2003451}
\bibitem{Chung2003451}
Chang-JoF. Chung and AndreaG. Fabbri.
\newblock Validation of spatial prediction models for landslide hazard mapping.
\newblock \emph{Natural Hazards}, 30\penalty0 (3):\penalty0 451--472, 2003.
\newblock ISSN 0921-030X.
%\newblock \doi{10.1023/B:NHAZ.0000007172.62651.2b}.
\newblock URL \url{http://dx.doi.org/10.1023/B\%3ANHAZ.0000007172.62651.2b}.  

%\bibitem[Lee et~al.(2004)Lee, Choi, and Min]{doi:10.1080/01431160310001618734}
\bibitem{doi:10.1080/01431160310001618734}
S.~Lee, J.~Choi, and K.~Min.
\newblock Probabilistic landslide hazard mapping using gis and remote sensing
  data at boun, korea.
\newblock \emph{International Journal of Remote Sensing}, 25\penalty0
  (11):\penalty0 2037--2052, 2004.
%\newblock \doi{10.1080/01431160310001618734}.
\newblock URL
  \url{http://www.tandfonline.com/doi/abs/10.1080/01431160310001618734}.
  
  
  

\bibitem{akgun2012}
Akgun, Aykut
\newblock A comparison of landslide susceptibility maps produced by logistic regression, multi criteria decision, and likelihood ratio methods: a case study at Izmir, Turkey
\newblock \emph{Landslides},
  (11):\penalty0 9, 1, 93--106, 2012
\newblock Springer. 


\bibitem{gaskill}
Gaskill, Jacob and Zuber, Brian and Nordman, Erik
\newblock Analyzing Landslide Susceptibility in St. Vincent and the Grenadines Using Co-Kriging and Logistic Regression
\newblock \emph{The 2015 IMAGIN Award, Michigan United States}.
\newblock URL
  \url{http://www.imagin.org/awards/sppc/2015/papers/jacob_gaskill_paper.pdf}.
  
\bibitem{garcia2008}
García-Rodríguez M.J., Malpica J.A., Benito B., Díaz M.
\newblock  Susceptibility assessment of earthquake-triggered landslides in El Salvador using logistic regression.
\newblock \emph{Geomorphology}
\penalty0 95, 172-191, 2008. 
  


\bibitem{garcia2010}
García-Rodríguez M.J., Malpica J.A.
\newblock Assessment of earthquake-triggered landslide susceptibility in El Salvador based on an Artificial Neural Network model.
\newblock \emph{Nat. Hazards Earth Syst. Sci.}
\penalty0 10, 1307–1315, 2010. 
\newblock doi:10.5194/nhess-10-1307-2010.
  

%\bibitem[Melchiorre et~al.(2011)Melchiorre, Abella, van Westen, and
%  Matteucci]{Melchiorre2011410}
\bibitem{Melchiorre2011410}
C.~Melchiorre, E.A.~Castellanos Abella, C.J. van Westen, and M.~Matteucci.
\newblock Evaluation of prediction capability, robustness, and sensitivity in
  non-linear landslide susceptibility models, guantánamo, cuba.
\newblock \emph{Computers {\&} Geosciences}, 37\penalty0 (4):\penalty0 410 --
  425, 2011.
\newblock ISSN 0098-3004.
%\newblock \doi{http://dx.doi.org/10.1016/j.cageo.2010.10.004}.
\newblock URL
  \url{http://www.sciencedirect.com/science/article/pii/S0098300410003286}.

%\bibitem[Zeng-wang(2001)]{Zeng2001374}
\bibitem{Zeng2001374}
Xu~Zeng-wang.
\newblock Gis and ann model for landslide susceptibility mapping.
\newblock \emph{Journal of Geographical Sciences}, 11\penalty0 (3):\penalty0
  374--381, 2001.
\newblock ISSN 1009-637X.
%\newblock \doi{10.1007/BF02892323}.
\newblock URL \url{http://dx.doi.org/10.1007/BF02892323}.


%\bibitem[Erm(2005)]{Ermini2005327}
\bibitem{Ermini2005327}
Artificial neural networks applied to landslide susceptibility assessment.
\newblock \emph{Geomorphology}, 66\penalty0 (1–4):\penalty0 327 -- 343, 2005.
\newblock ISSN 0169-555X
%\newblock \doi{http://dx.doi.org/10.1016/j.geomorph.2004.09.025}.
\newblock URL \url{http://www.sciencedirect.com/science/article/pii/S0169555X04002272}.



\bibitem{Yesilnacar2005251}
E. Yesilnacar and T. Topal.
\newblock Landslide susceptibility mapping: A comparison of logistic regression and neural networks methods in a medium scale study, Hendek region (Turkey).
\newblock \emph{Engineering Geology}, 79\penalty0
  (11):\penalty0 251-266, 2005.
\newblock ISSN 0013-7952,
%\newblock \doi{10.1080/01431160310001618734}.
\newblock URL
  \url{http://www.sciencedirect.com/science/article/pii/S0013795205000384}.









\bibitem{ballabio2012support}
Ballabio, Cristiano and Sterlacchini, Simone
\newblock Support vector machines for landslide susceptibility mapping: the Staffora River Basin case study, Italy
\newblock \emph{Mathematical geosciences},
  (11):\penalty0 1, 47--70, 2012.
\newblock Springer. 



\bibitem{tien2012landslide}
Tien Bui, Dieu and Pradhan, Biswajeet and Lofman, Owe and Revhaug, Inge
\newblock Landslide susceptibility assessment in vietnam using support vector machines, decision tree, and Naive Bayes Models
\newblock \emph{Mathematical Problems in Engineering},
  (11):\penalty0 2012, 2012.
\newblock Hindawi Publishing Corporation.




\bibitem{van2006landslide}
Van Westen, CJ and Van Asch, Th WJ and Soeters, R
\newblock Landslide hazard and risk zonation why is it still so difficult?
\newblock \emph{Bulletin of Engineering geology and the Environment}.
(65):\penalty0 2, 167--184, 2006
\newblock Springer. 


%%%%%%%%%%%%%%  Study Area %%%%%%%%%%%%%%%%%%%%%%%%%%%%%%
%%%%%%%%%%%%%%%%%%%%%%%%%%%%%%%%%%%%%%%%%%%%%%%%%%%%%%%%%



\bibitem{minecon}
Ministerio de Economia- Direcci\'{o}n General de Estad\'{i}stica y Censos
\newblock VI Censo de Población y V de Vivienda
\newblock \emph{2007-2008}.
\penalty0 Available from http://goo.gl/yvjzPy




\bibitem{schmidt1975}
Schmidt-Thom\'{e} M.
\newblock The geology in San Salvador area (El Salvador, Central America), a basis for city development and planning.
\newblock \emph{Geol. Jb.}.
\penalty0 13, 207-228, 1975.


\bibitem{chavez2014a}
Ch\'{a}vez J.A., Sebesta J., Kopecky L., L\'{o}pez R.
\newblock Application of Geomorphologic Knowledge for Erosion Hazard Mapping
\newblock \emph{Natural Hazards}.
\penalty0 71,1323–1354, 2014.
\newblock DOI 10.1007/s11069-013-0948-8.


\bibitem{hernan2004}
Hern\'{a}ndez W.E.
\newblock Características Geomec\'{a}nicas y Vulcanologicas de las Tefras Tierra Blanca Joven, Caldera de Ilopango, El Salvador.
\newblock MSc Thesis. Universidad Politécnica de Madrid – Universidad Polit\'{e}cnica de El Salvador.
\penalty0 2004


\bibitem{chavez2014b}
Chávez J.A.,  Landaverde J.M., Ayala O.E., Mendoza L.E. (2014b). 
\newblock Application of Constitutive Models in the Volcanic Tephra ``Tierra Blanca Joven''.
\newblock \emph{Ingeniería.}
\penalty0  24 (2).53-78.
\newblock  ISSN: 2215-2652.


\bibitem{rolo2004}
Rolo R., Bommer J.J., Houghton B.F., Vallance J.W., Berdousis P., Mavrommati C., Morphy W.
\newblock Geologic and engineering characterization of Tierra Blanca pyroclastic ash deposits.
\newblock \emph{Chapter 5 in Geological Society of America Special paper 375 in Natural Hazards in El Salvador of Rose W.I., Bommer J.J., López D.L., Carr M.J. and Major J.J.}
\penalty0  55-67, 2004.


\bibitem{rymer1987}
Rymer M.J. 
\newblock The San Salvador Earthquake of October 10, 1986.
\newblock \emph{Geologic Aspects Earthquake Spectra.}
\penalty0 3(3), 435-463, 1987.


\bibitem{dull2004}
Dull R.A.
\newblock Lessons from the mud, lessons from the Maya: Paleoecological records of the Tierra Blanca Joven eruption.
\newblock \emph{Chapter 7 in Geological Society of America Special paper 375 in Natural Hazards in El Salvador of Rose W.I., Bommer J.J., López D.L., Carr M.J. and Major J.J.}
\penalty0 237-244., 2004.

\bibitem{major2004}
Major J.J. Chilling S.P., Pullinguer C.R., Escobar C.D.
\newblock Debris-flow hazards at San Salvador, San Vicente, and San Miguel volcanoes, El Salvador.
\newblock \emph{Chapter 7 in Geological Society of America Special paper 375 in Natural Hazards in El Salvador of Rose W.I., Bommer J.J., López D.L., Carr M.J. and Major J.J.}
\penalty0 89-108, 2004.


\bibitem{sofield2004}
Sofield D.
\newblock Eruptive history and volcanic hazards of Volcan San Salvador. 
\newblock \emph{Chapter 7 in Geological Society of America Special paper 375 in Natural Hazards in El Salvador of Rose W.I., Bommer J.J., López D.L., Carr M.J. and Major J.J.}
\penalty0 147-158, 2004.






%%%%%%%%%%%%%%%%%%%%%%%%%%%%%%%%%%%%%%%%%%%%%%%%%%%%%%%%%%
%%%%%%%%%%%%%%%%%%%%%%%%%%%%%%%%%%%%%%%%%%%%%%%%%%%%%%%%%%
%%%%%%%%%%%%%%%%   Landslide inventory and data sources of input variables                        %%%%%%%%%%%%%%%
%%%%%%%%%%%%%%%%%%%%%%%%%%%%%%%%%%%%%%%%%%%%%%%%%%%%%%%%%%
%%%%%%%%%%%%%%%%%%%%%%%%%%%%%%%%%%%%%%%%%%%%%%%%%%%%%%%%%%



\bibitem{beni2012}
Benito M.B., Lindholm C., Camacho E., Climent Á., Marroquín G., Molina E.,Rojas W., Escobar J.J., Talavera E., Alvarado G.E. and Torres Y.
\newblock A New Evaluation of Seismic Hazard for the Central America Region. 
\newblock \emph{Bulletin of the Seismological Society of America}
\penalty0 102 (2), 504-523. 2012.
\newblock  doi: 10.788/0120110015






%%%%%%%%%%%%%%%%%%%%%%%%%%%%%%%%%%%%%%%%%%%%%%%%%%%%%%%%%%%%%%%%%
%%%%%%%%%%%%%%%%%%%%%%%%%%%%%%%%%%%%%%%%%%%%%%%%%%%%%%%%%%%%%%%%%



\bibitem{McNelis2005}
Paul D. McNelis
\newblock Estimation of a Network with Evolutionary Computation
\newblock \emph{Academic Press},
  (11):\penalty0 59-84, 2005.
\newblock ISBN 9780124859678,
%\newblock \doi{http://dx.doi.org/10.1016/B978-012485967-8.50003-8}.
\newblock URL
  \url{http://www.sciencedirect.com/science/article/pii/B9780124859678500038}.



\bibitem{FAQANN}
Warren S. Sarle
\newblock URL
  \url{ftp://ftp.sas.com/pub/neural/FAQ.html#questions}.
  \newblock Accessed 31 May, 2015.



%%%%%%%%%%%%%%%%%%%%%%%%%%%%%%%%%%%%%%%%%%%%%%%%%%%%%%%%%
%%%%%%%%%%%%%%  Discussion  %%%%%%%%%%%%%%%%%%%%%%%%%%%%%
%%%%%%%%%%%%%%%%%%%%%%%%%%%%%%%%%%%%%%%%%%%%%%%%%%%%%%%%%%

\bibitem{fernan2008}
Fernandez-Lavado C., Sanchez A., Amenos M., Barrio J.
\newblock Caracterización de la susceptibilidad y de la amenaza por movimientos de ladera del Área Metropolitana de San Salvador (AMSS). Scale 1: 75000 (1 Sheet).
\newblock \emph{Project IPGARAMSS framework. Geólogos del Mundo. San Salvador, El Salvador.}
\penalty0 2008.




\bibitem{snet2004}
SNET-MARN, Servicio Nacional de Estudios Territoriales-Ministerio de Ambiente y Recursos Naturales
\newblock Memoria Técnica Para El Mapa De Susceptibilidad de Deslizamientos de Tierra En El Salvador.
\newblock \emph{Ministerio de Ambiente y Recursos Naturales}
\penalty0 [cited 2015 June 04]. Map available from http://www.snet.gob.sv/ver/geologia/susceptibilidad+actual/


\bibitem{aki1993}
Aki, K. 
\newblock Local site effects on weak and strong ground motion. In F. Lund (ed.) New Horizons in Strong Motion: Seismic Studies and Engineering Practice.
\newblock \emph{Tectonophysics}
\penalty0 218, 93-111, 1993.   


\bibitem{crosta2005}
Crosta G.B., Imposimato S., Rodderman, Chiesa S.m, Moia F. 
\newblock Small fast-moving flow-like landslides in volcanic deposits: The 2001 Las Colinas Landslide (El Salvador). 
\newblock \emph{Engineering Geology.}
\penalty0 79, 185-214, 2005.










%%%%%%%%%%%%%%%%%%%%%%%%%%%%%%%%%%%%%%%%%%%%%%%%%%%%%%%%%
%%%%%%%%%%%%%%%%%%%%%%%%%%%%%%%%%%%%%%%%%%%%%%%%%%%%%%%%%
%%%%%%%%%%%%%%%%%%%%%%%%%%%%%%%%%%%%%%%%%%%%%%%%%%%%%%%%%%




\end{thebibliography}

\end{document}

